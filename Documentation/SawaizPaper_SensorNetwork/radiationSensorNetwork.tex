%%
%%  Manuscript for Quark Matter 2014 Proceedings
%% ====================================
%% Author: 
%% Collaboration:
%% Last updated:
%%
%% This is a basic template. For more detailed options
%% see http://www.elsevier.com/wps/find/authorsview.authors/latex

%-----------------------------------------------------------------------------------

%% This template uses the elsarticle.cls document class and the extension package ecrc.sty
%% as well as the logo files elsevier-logo-3p.pdf, SDlogo-3p.pdf. 
%% For full documentation on usage of elsarticle.cls, consult the documentation "elsdoc.pdf"
%% Further resources available at http://www.elsevier.com/latex

%-----------------------------------------------------------------------------------

%%%%%%%%%%%%%%%%%%%%%%%%%%%%%%%%%%%%%%%%%%%%%%
%%%%%%%%%%%%%%%%%%%%%%%%%%%%%%%%%%%%%%%%%%%%%%
%%                                          %%
%% Important note on usage                  %%
%% -----------------------                  %%
%% This file must be compiled with PDFLaTeX %%
%% Using standard LaTeX will not work!      %%
%%                                          %%
%%%%%%%%%%%%%%%%%%%%%%%%%%%%%%%%%%%%%%%%%%%%%%
%%%%%%%%%%%%%%%%%%%%%%%%%%%%%%%%%%%%%%%%%%%%%%

%% The '3p' and 'times' class options of elsarticle are used for Elsevier CRC
\documentclass[3p,times]{elsarticle}

%% The `ecrc' package must be called to make the CRC functionality available
\usepackage{ecrc}

%% The ecrc package defines commands needed for running heads and logos.
%% For running heads, you can set the journal name, the volume, the starting page and the authors

%% If you have eps figures, use the epstopdf package
\usepackage{epstopdf}

%% set the volume if you know. Otherwise `00'
\volume{00}

%% set the starting page if not 1
\firstpage{1}

%% Give the name of the journal
\journalname{Nuclear Physics A}

%% Give the author list to appear in the running head
%% Example \runauth{C.V. Radhakrishnan et al.}
\runauth{Author1 et al.}

%% The choice of journal logo is determined by the \jid and \jnltitlelogo commands.
%% A user-supplied logo with the name <\jid>logo.pdf will be inserted if present.
%% e.g. if \jid{yspmi} the system will look for a file yspmilogo.pdf
%% Otherwise the content of \jnltitlelogo will be set between horizontal lines as a default logo

%% Give the abbreviation of the Journal.
\jid{nupha}

%% Give a short journal name for the dummy logo (if needed)
\jnltitlelogo{Nuclear Physics A}

%% Hereafter the template follows `elsarticle'.
%% For more details see the existing template files elsarticle-template-harv.tex and elsarticle-template-num.tex.

%% Elsevier CRC generally uses a numbered reference style
%% For this, the conventions of elsarticle-template-num.tex should be followed (included below)
%% If using BibTeX, use the style file elsarticle-num.bst

%% End of ecrc-specific commands
%%%%%%%%%%%%%%%%%%%%%%%%%%%%%%%%%%%%%%%%%%%%%%%%%%%%%%%%%%%%%%%%%%%%%%%%%%

%% Useful packages
\usepackage{graphicx}
\usepackage{amsmath,amssymb}
%% The amssymb package provides various useful mathematical symbols
%% \usepackage{amssymb}
%% The amsthm package provides extended theorem environments
%% \usepackage{amsthm}

%% The lineno packages adds line numbers. Start line numbering with
%% \begin{linenumbers}, end it with \end{linenumbers}. Or switch it on
%% for the whole article with \linenumbers after \end{frontmatter}.
%% \usepackage{lineno}

%% natbib.sty is loaded by default. However, natbib options can be
%% provided with \biboptions{...} command. Following options are
%% valid:

%%   round  -  round parentheses are used (default)
%%   square -  square brackets are used   [option]
%%   curly  -  curly braces are used      {option}
%%   angle  -  angle brackets are used    <option>
%%   semicolon  -  multiple citations separated by semi-colon
%%   colon  - same as semicolon, an earlier confusion
%%   comma  -  separated by comma
%%   numbers-  selects numerical citations
%%   super  -  numerical citations as superscripts
%%   sort   -  sorts multiple citations according to order in ref. list
%%   sort&compress   -  like sort, but also compresses numerical citations
%%   compress - compresses without sorting
%%
%% \biboptions{comma,round}

% \biboptions{}

% if you have landscape tables
%\usepackage[figuresright]{rotating}

% put your own definitions here:
%   \newcommand{\cZ}{\cal{Z}}
%   \newtheorem{def}{Definition}[section]
%   ...

% add words to TeX's hyphenation exception list
%\hyphenation{author another created financial paper re-commend-ed Post-Script}

% declarations for front matter

\begin{document}

\begin{frontmatter}

%% Title, authors and addresses

%% use the tnoteref command within \title for footnotes;
%% use the tnotetext command for the associated footnote;
%% use the fnref command within \author or \address for footnotes;
%% use the fntext command for the associated footnote;
%% use the corref command within \author for corresponding author footnotes;
%% use the cortext command for the associated footnote;
%% use the ead command for the email address,
%% and the form \ead[url] for the home page:
%%
%% \title{Title\tnoteref{label1}}
%% \tnotetext[label1]{}
%% \author{Name\corref{cor1}\fnref{label2}}
%% \ead{email address}
%% \ead[url]{home page}
%% \fntext[label2]{}
%% \cortext[cor1]{}
%% \address{Address\fnref{label3}}
%% \fntext[label3]{}

\title{Low-cost Radiation Sensor Network}


%% Single author (and collaboration) - please insert
\author{Sawaiz Syed and Xiaochun He}
%\fntext[col1] {A list of members of the XYZ Collaboration and acknowledgements can be found at the end of this issue.}
\address{Department of Physics and Astronomy, Georgia State University, Atlanta, GA 30303}

%% For multiple authors, replace the above by:

%\author[label1]{Author1}
%\author[label2]{Author2}

%\address[label1]{Address 1}
%\address[label2]{Address 2}

\begin{abstract}
%% Text of abstract
A novel radiation sensor network has been developed for monitoring radioactive events in real-time.  The sensor nodes consists of low-cost Geiger tubes which are read out using AVR microcontroller with a 2.4 GHz wireless interface.  The receiver node is implemented with a Raspberry PI which provides the database interface and the network with other servers.  
\end{abstract}

\begin{keyword}
%% keywords here, in the form: keyword \sep keyword
Keyword1 \sep Keyword2 \sep Keyword3
%% MSC codes here, in the form: \MSC code \sep code
%% or \MSC[2008] code \sep code (2000 is the default)

\end{keyword}

\end{frontmatter}

%%
%% Start line numbering here if you want
%%
% \linenumbers

%% main text

\section{Introduction}
\label{intro}


\section{Sensor Node Design and Implementation}

A sensor node consists of a radiation sensor (i.e. Geiger tube), a power supply unit and readout circuit, a microcontroller and a wireless transceiver as shown in Figure~\ref{fig:sensorNode}.  The microcontroller is a 8-bit RISC-based AVR (ATtiny2313) made by Atmel and is programed in C for controlling Geiger tube high voltage, signal readout and wireless communication.
\begin{figure}[h]
\begin{center}
%\includegraphics*[width=9.cm]{bessel1}\\
\includegraphics*[width=9.cm]{SchematicImages/Assembly.jpg}
\caption{
Sensor node assembly}
\label{fig:sensorNode}
\end{center}
\end{figure}

In the current design, we pick a Geiger tube (SBM-20) made by VNIITFA company in Russia. The stainless steel Geiger tube is filled with a mixture of Ne + Br$_2$ + Ar gases and has a dimension of 108 mm in length and 11 mm in diameter. Its operating high voltage is 350 - 475 volts (the recommended voltage is 400 volts). The tube life is about at least $2\times10^{10}$ pulses and has a dead time of 190 $\mu$s.  The Geiger tube high voltage power supply is deigned as a DC-to-DC boost converter as shown in Figure~\ref{fig:GTpower} for reducing both the cost and physical dimensions. The high voltage output can be program with the onboard microcontroller (ATtiny2313). 
%\subsection{Geiger-tube }
\begin{figure}[h]
\begin{center}
%\includegraphics*[width=9.cm]{bessel1}\\
\includegraphics*[width=9.cm]{SchematicImages/HV_Supply.jpg}
\caption{
High voltage power supply circuit design for the Geiger tube.}
\label{fig:GTpower}
\end{center}
\end{figure}
%

\begin{figure}[h]
\begin{center}
%\includegraphics*[width=9.cm]{bessel1}\\
\includegraphics*[width=8.cm]{SchematicImages/Signal.jpg}
\caption{
Digital signal output}
\label{fig:readout}
\end{center}
\end{figure}
%

%

\begin{figure}
\begin{center}
%\includegraphics*[width=9.cm]{bessel1}\\
\includegraphics*[width=9.cm]{SchematicImages/ICSP.jpg}
\caption{Header}
\label{fig:generic}
\end{center}
\end{figure}
%


\begin{figure}
\begin{center}
%\includegraphics*[width=9.cm]{bessel1}\\
\includegraphics*[width=10.cm]{SchematicImages/NRF24L01.jpg}
\caption{Wireless module interface circuit.}
\label{fig:generic}
\end{center}
\end{figure}

\section{Receiver Node Design and Implementation}
The receiver node consists of the same wireless transceiver and a host computer (i.e., Raspberry PI). 
\begin{figure}
\begin{center}
%\includegraphics*[width=9.cm]{bessel1}\\
\includegraphics*[width=8.cm]{SchematicImages/RPI_Wireless.jpg}
\caption{Wireless module interface with Raspberry PI.}
\label{fig:generic}
\end{center}
\end{figure}


\section{Sensor Network Integration}

\section{Results and Summary}


\clearpage

%% The Appendices part is started with the command \appendix;
%% appendix sections are then done as normal sections
%% \appendix

%% \section{}
%% \label{}

%% References
%%
%% Following citation commands can be used in the body text:
%% Usage of \cite is as follows:
%%   \cite{key}         ==>>  [#]
%%   \cite[chap. 2]{key} ==>> [#, chap. 2]
%%

%% References with BibTeX database:

%\bibliographystyle{elsarticle-num}
%\bibliography{<your-bib-database>}

%% Authors are advised to use a BibTeX database file for their reference list.
%% The provided style file elsarticle-num.bst formats references in the required Procedia style

%% For references without a BibTeX database:

\begin{thebibliography}{00}

%% \bibitem must have the following form:
%%   \bibitem{key}...
%%

\bibitem{ref1} J. van der Geer, J.A.J. Hanraads, R.A. Lupton, J. Sci. Commun. 163 (2000) 51�59. 
\bibitem{ref2} W. Strunk Jr., E.B. White, The Elements of Style, third ed., Macmillan, New York, 1979. 

\end{thebibliography}


\clearpage

\appendix{Schematic Diagrams}

\begin{figure}
\begin{center}
%\includegraphics*[width=9.cm]{bessel1}\\
\includegraphics*[width=10.cm]{SchematicImages/MCU.jpg}
\caption{Atmel Microcontroller of the sensor node.}
\label{fig:microcontroller}
\end{center}
\end{figure}

\end{document}

%%
%% End of file `ecrc-template.tex'. 